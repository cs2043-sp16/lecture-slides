% Cornell CS 2043: Unix Tools and Scripting
% Spring 2016
%
% http://cs2043-sp16.github.io/
%
% These slides have been created using the metroplis theme demo found here:
%
%     https://github.com/matze/mtheme.git
%
% I am a huge fan of the metropolis theme...thanks @matze!
\input{../common-header.input}

%%%%%%%%%%%%%%%%%%%%%%%%%%%%%%%%%%%%%%%%%%%%%%%%%%%%%%%%%%%%%%%%%%%%%%%%%%%%%%%%%%%%%%%%%%%%%%%%%%%%%%%%%%%%%%%
%%%%%%%%%%%%%%%%%%%%%%%%%%%%%%%%%%%%%%%%%%%%%%%%%%%%%%%%%%%%%%%%%%%%%%%%%%%%%%%%%%%%%%%%%%%%%%%%%%%%%%%%%%%%%%%
%%%%%%%%%%%%%%%%%%%%%%%%%%%%%%%%%%%%%%%%%%%%%%%%%%%%%%%%%%%%%%%%%%%%%%%%%%%%%%%%%%%%%%%%%%%%%%%%%%%%%%%%%%%%%%%
%%%%%%%%%%%%%%%%%%%%%%%%%%%%%%%%%%%%%%%%%%%%%%%%%%%%%%%%%%%%%%%%%%%%%%%%%%%%%%%%%%%%%%%%%%%%%%%%%%%%%%%%%%%%%%%
%
% \begin
%
%%%%%%%%%%%%%%%%%%%%%%%%%%%%%%%%%%%%%%%%%%%%%%%%%%%%%%%%%%%%%%%%%%%%%%%%%%%%%%%%%%%%%%%%%%%%%%%%%%%%%%%%%%%%%%%
%%%%%%%%%%%%%%%%%%%%%%%%%%%%%%%%%%%%%%%%%%%%%%%%%%%%%%%%%%%%%%%%%%%%%%%%%%%%%%%%%%%%%%%%%%%%%%%%%%%%%%%%%%%%%%%
%%%%%%%%%%%%%%%%%%%%%%%%%%%%%%%%%%%%%%%%%%%%%%%%%%%%%%%%%%%%%%%%%%%%%%%%%%%%%%%%%%%%%%%%%%%%%%%%%%%%%%%%%%%%%%%
%%%%%%%%%%%%%%%%%%%%%%%%%%%%%%%%%%%%%%%%%%%%%%%%%%%%%%%%%%%%%%%%%%%%%%%%%%%%%%%%%%%%%%%%%%%%%%%%%%%%%%%%%%%%%%%
\title{07 \-- Processes and Jobs}
\subtitle{CS 2043: Unix Tools and Scripting, Spring 2016 \cite{prevSemesters}}
\date{February 10th, 2016}
\author{Stephen McDowell}
\institute{Cornell University}

\begin{document}
\maketitle

% TOC
\begin{frame}{Table of contents}
  \setbeamertemplate{section in toc}[sections numbered]
  \tableofcontents[hideallsubsections]
\end{frame}

\begin{frame}{Some Logistics}
  \begin{itemize}[<+- | alert@+>]
    \item HW1 due Friday, 2/12/2016 at 5pm
    \item Drop deadline is \colbf{today}.
    \item Lecture-demo solutions...thanks Joe!
    \item The nature of the material in this topic basically dictates not covering OSX.  They may exist, they
          may not.
    \begin{itemize}[<+- | alert@+>]
      \item They may also give very different results.
    \end{itemize}
  \end{itemize}
\end{frame}

%
%%
%%%
%%%%
%%%%% section processes_overview
%%%%%
%%%%%
%%%%%
%%%%%
\section{Processes Overview}
\label{sec:processes_overview}

\begin{frame}[fragile]{What is a Process?}
  \begin{itemize}[<+- | alert@+>]
    \item A process is just an instance of a running program.
    \item Not just a "program" \-- it is being \emph{executed}.
    \item Not just a "running program", as you can execute the same program multiple times.
    \begin{itemize}[<+- | alert@+>]
      \item These would be multiple processes running an instance of the same program.
    \end{itemize}
    \item Example: if you open more than one terminal (windows or tabs), you are running multiple processes
          of your shell.
  \end{itemize}
\end{frame}

\begin{frame}[fragile]{Identification}
  \begin{itemize}[<+- | alert@+>]
    \item Processes have a unique "Process ID" (PID) when created.
    \item The PID allows you to distinguish between multiple instances of the same program.
    \item There are countless ways to discover the PID, as well as what processes are running.
    \item These methods often depend on how much information you want, as well as what your user priviliges are.
  \end{itemize}
\end{frame}

\begin{frame}[fragile]{Identification: \texttt{ps}}
  \begin{block}{\colbf{P}rocess \colbf{S}napshot}
    \texttt{ps [options]}
    \begin{itemize}
      \item Reports a snapshot of the current running processes, including PIDs.
      \item By default, only the processes started by the user.
      \item \texttt{-e}: lists every process currently running on the system
      \item \texttt{-ely}: Gives more information than you can handle.
      \item \texttt{-u <username>}: lists all processes for user \texttt{username}.
      \item \colbf{Note}: very different for BSD/OSX, read the man page...
    \end{itemize}
  \end{block}
  \vspace*{-2ex}
  \begin{itemize}[<+- | alert@+>]
    \item To see more information about a process, pipe through \texttt{grep}.
    \item For example: \texttt{ps -e | grep firefox} shows us the results about \texttt{firefox} processes.
  \end{itemize}
\end{frame}

\begin{frame}[fragile]{Identification: \texttt{lsof}}
  \begin{block}{\colbf{L}i\colbf{s}t of \colbf{O}pen \colbf{F}iles}
    \texttt{lsof [options]}
    \begin{itemize}
      \item Very similar to \texttt{ps}, with more information by default.
      \item Frequently used for monitoring port connections...
      \item \texttt{-i}: lists IP sockets
      \begin{itemize}
        \item \texttt{lsof -i tcp:843} shows all tcp processes on port \texttt{843}
      \end{itemize}
      \item Many options...read the man page if you are intrigued.
    \end{itemize}
  \end{block}
  \begin{itemize}[<+- | alert@+>]
    \item As with \texttt{ps}, often best served with a side of \texttt{grep}.
    \item More useful for administration, especially when managing a networked environment.
  \end{itemize}
\end{frame}

\begin{frame}[fragile]{Resource Usage}
  \begin{block}{Display and Update \colbf{top} CPU Processes}
    \texttt{top [options]}
    \begin{itemize}
      \item Displays the amount of resources in percentages each process is using.
      \item \texttt{-d <seconds>}: control update frequency
      \begin{itemize}
        \item The act of monitoring is an expensive process...
      \end{itemize}
      \item \texttt{-u <user>}: only show processes owned by \texttt{user}
      \item \texttt{-p <PID>}: show statistics on process with id \texttt{PID} only.
    \end{itemize}
  \end{block}
  \begin{itemize}[<+- | alert@+>]
    \item When used in conjunction with \texttt{ps} or \texttt{lsof}, can be a very powerful analysis tool.
    \item Example sequence on the next page.
  \end{itemize}
\end{frame}

\begin{frame}[fragile]{Example: Resource Monitoring}
  \begin{minted}[bgcolor=bg, gobble=4, fontsize=\fontsize{6.5}{0}]{bash}
    > ps -e | grep firefox
    12975 ?        00:01:45 firefox
    > top -p 12795
    top - 09:37:56 up 1 day, 13:52,  5 users,  load average: 0.19, 0.20, 0.19
    Tasks:   1 total,   0 running,   1 sleeping,   0 stopped,   0 zombie
    %Cpu(s):  1.1 us,  0.5 sy,  0.0 ni, 98.4 id,  0.0 wa,  0.0 hi,  0.0 si,  0.0 st
    KiB Mem : 16386660 total,  5990760 free,  3562320 used,  6833580 buff/cache
    KiB Swap:  4194300 total,  4194300 free,        0 used. 12551476 avail Mem

      PID USER      PR  NI    VIRT    RES    SHR S  %CPU %MEM     TIME+ COMMAND
    12975 sven      20   0 1437888 396868 105116 S   1.7  2.4   1:46.39 firefox
  \end{minted}
  \begin{itemize}[<+- | alert@+>]
    \item You'll be best off reading through the man page to understand everything going on here.
    \item Some great examples in \cite{top}.
    \begin{itemize}[<+- | alert@+>]
      \item I've found myself on that website \emph{many} times, he has a lot of excellent examples about
            a large quantity of topics.
    \end{itemize}
  \end{itemize}
\end{frame}

\begin{frame}[fragile]{Example: Resource Monitoring}
  \begin{itemize}[<+- | alert@+>]
    \item Now I have opened about thirty tabs in firefox, and we get much different results:
    \item Look at the cpu usage
      \begin{minted}[bgcolor=bg, gobble=8, fontsize=\fontsize{6.5}{0}]{bash}
        > top -p 12795
        top - 09:43:09 up 1 day, 13:57,  5 users,  load average: 1.33, 0.75, 0.41
        Tasks:   1 total,   1 running,   0 sleeping,   0 stopped,   0 zombie
        %Cpu(s): 13.4 us,  3.3 sy,  0.0 ni, 83.2 id,  0.0 wa,  0.0 hi,  0.0 si,  0.0 st
        KiB Mem : 16386660 total,  3622768 free,  5679500 used,  7084392 buff/cache
        KiB Swap:  4194300 total,  4194300 free,        0 used. 10300816 avail Mem

          PID USER      PR  NI    VIRT    RES    SHR S  %CPU %MEM     TIME+ COMMAND
        12975 sven      20   0 3451396 1.372g 133688 R  75.7  8.8   5:00.96 firefox
      \end{minted}
    \item 75.7\%?!!! Pretty common actually, this is why I always tell you to use your browser inside your
          Virtual Machine...
  \end{itemize}
\end{frame}

%%%%%
%%%%%
%%%%%
%%%%%
%%%%% section processes_overview
%%%%
%%%
%%
%

%
%%
%%%
%%%%
%%%%% section modifying_processes
%%%%%
%%%%%
%%%%%
%%%%%
\section{Modifying Processes}
\label{sec:modifying_processes}

\begin{frame}[fragile]{Priority}
  \begin{itemize}[<+- | alert@+>]
    \item Suppose you want to run some long calculation that might take days, but would consume 100\% of your
          CPU.
    \item Can we tell the server to give your process less priority in terms of CPU time?
    \item Recall that although Unix seems to run tens or hundreds of processes at once, one CPE can only run
          one process at a time\tsup{*}.
    \item Quick switching back and forth between processes makes it seem as though they are all running simultaneously.
    \item The Unix masters anticipated this need, and each process was given a \texttt{priority} when it starts.
  \end{itemize}
\end{frame}

\begin{frame}[fragile]{Initial Priority}
  Start a process with a non-default priority:
  \begin{block}{The \colbf{nice} command}
    \texttt{nice [options] command}
    \begin{itemize}
      \item Runs \texttt{command} with a specified \emph{"niceness"} value (default: 10)
      \item \emph{Niceness} values range from \texttt{-20} (highest priority) to \texttt{19} (lowest priority)
      \item Only \texttt{root} can give a process a \emph{negative niceness} value
      \item Commands run without \texttt{nice} have priority \texttt{0}.
    \end{itemize}
  \end{block}
  \begin{block}{Example}
    \texttt{nice -n 10 deluge}
    \begin{itemize}
      \item Keeps torrents from hogging the CPU.
    \end{itemize}
  \end{block}
\end{frame}

\begin{frame}[fragile]{Adjusting Priority}
  \begin{block}{The \colbf{renice} command}
    \texttt{renice <priority> -p <PID>}
    \begin{itemize}
      \item Changes the \emph{niceness} of the process with id \texttt{PID} to \texttt{<priority>}
      \item Remember: only \texttt{root} can assign \emph{negative} values.
      \item You can only \texttt{renice} a process you started.
    \end{itemize}
  \end{block}
  \begin{block}{Some Examples}
    \texttt{renice 5 -p 10275}
    \begin{itemize}
      \item Set the \emph{niceness} of the process with \texttt{PID 10275} to \texttt{5}
      \begin{itemize}
        \item Slightly lower than normal \emph{niceness}
      \end{itemize}
    \end{itemize}
    \texttt{renice 19 -u sven}
    \begin{itemize}
      \item Set the \emph{niceness} of \colbf{all} my processes to 19
    \end{itemize}
  \end{block}
\end{frame}

\begin{frame}[fragile]{Ending Processes: I}
  Sometimes you need to end a process.
  \begin{block}{\colbf{kill}}
    \texttt{kill [-signal] <PID>}
    \begin{itemize}
      \item Sends the specified \texttt{signal} to the process with id \texttt{PID}
      \item By default, it terminates execution
    \end{itemize}
  \end{block}
  \begin{block}{\colbf{killall}}
    \texttt{killall [-signal] <name>}
    \begin{itemize}
      \item Kills processes by name.
      \item E.g. \texttt{killall firefox}
    \end{itemize}
  \end{block}
  \colbf{Note}: These are dangerous commands, and should generally be last resorts.
\end{frame}

\begin{frame}[fragile]{Useful Kill Signals}
  \begin{itemize}[<+- | alert@+>]
    \item Kill signals can be used by number or name.
    \item \texttt{TERM} or \texttt{15}: terminates execution (default).
    \item \texttt{HUP} or \texttt{1}: hang-up (restarts the program)
    \item \texttt{KILL} or \texttt{9}: like bleach, can kill anything.
    \item Some examples:
    \begin{block}{Killing 101}
      \begin{itemize}
        \item \texttt{kill 9009}: terminates process with \texttt{PID 9009}
        \item \texttt{kill -9 3223}: \colbf{REALLY} kills the process with \texttt{PID 3223}
        \item \texttt{kill -HUP 12221}: restarts the process with \texttt{PID 12221}
        \begin{itemize}
          \item very useful for servers and daemon processes
        \end{itemize}
      \end{itemize}
    \end{block}
    \item Remember \texttt{top}? You can both \emph{renice} and \emph{kill} processes from within it!
  \end{itemize}
\end{frame}

%%%%%
%%%%%
%%%%%
%%%%%
%%%%% section modifying_processes
%%%%
%%%
%%
%

%
%%
%%%
%%%%
%%%%% section jobs
%%%%%
%%%%%
%%%%%
%%%%%
\section{Jobs}
\label{sec:jobs}

\begin{frame}[fragile]{What are Jobs?}
  \begin{block}{Jobs}
    A job is a process running \emph{under the influence} of a job control facility.
    \begin{itemize}
      \item Job control is a built-in feature of most shells, allowing the user to pause and resume tasks.
      \item The user can also run them in the background.
      \item Not covered here: \texttt{crontab}.  For the future sys admins, read the article in \cite{crontab}
    \end{itemize}
  \end{block}
\end{frame}

\begin{frame}[fragile]{Why do you want this?}
  Let's use \texttt{ping} as an example.
  \begin{block}{\colbf{Ping}}
    \texttt{ping <server>}
    \begin{itemize}
      \item Measures network response time (latency) to a remote server.
      \item Sends short bursts to the server, then measures time until they return.
    \end{itemize}
  \end{block}
  \begin{block}{Example:}
    \texttt{ping google.com}
    \begin{itemize}
      \item Remember, \texttt{ctrl+c} kills the process.
    \end{itemize}
  \end{block}
\end{frame}

\begin{frame}[fragile]{Why we Need Job Control}
  As long as \texttt{ping} runs, we lose control of our shell.  This happens with many other applications.
  \begin{itemize}[<+- | alert@+>]
    \item Moving large quantities of files
    \item Compiling source code
    \item Playing multimedia
    \item Scientific computing
    \item etc
    \begin{block}{Example:}
      \texttt{vlc }
    \end{block}
  \end{itemize}
\end{frame}

\begin{frame}[fragile]{Starting a Job in the Background}
  To run a job in the background, we will use a new operator:
  \begin{block}{\colbf{\&}}
    \texttt{<command> [arguments] \&}
    \begin{itemize}
      \item Runs the specified command as a background job.
      \item Unless told otherwise, will send output to the terminal!
      \item But at least we can type in our terminal again.
    \end{itemize}
  \end{block}
  \begin{block}{Example:}
    \texttt{vlc best\_song\_ever.flac \&}
  \end{block}
\end{frame}

\begin{frame}[fragile]{Sending a Job to the Background}
  If you already started the job, but don't want to wait any more:

  \begin{block}{Pausing a Job}
    Press \texttt{ctrl+z} to pause a running process!
  \end{block}

  \begin{itemize}[<+- | alert@+>]
    \item Note this is still \texttt{ctrl} even on Mac...just like \texttt{ctrl+c}
    \item The shell will pause the jobs \texttt{JOB ID} (similar to \texttt{PID})
    \item We can bring it back.
  \end{itemize}
\end{frame}

\begin{frame}[fragile]{Revivals}
  \begin{block}{\colbf{B}ack\colbf{g}round}
    \texttt{bg <JOB ID>}
    \begin{itemize}
      \item Resumes the job with id \texttt{JOB ID} in the \emph{background}
      \item Without \texttt{JOB ID}, resumes last job placed in background.
    \end{itemize}
  \end{block}
  \vspace*{-1ex}
  \begin{block}{\colbf{F}ore\colbf{g}round}
    \texttt{fg <JOB ID>}
    \begin{itemize}
      \item Resumes the job with id \texttt{JOB ID} in the \emph{foreground}.
      \item Without \texttt{JOB ID}, resumes last job placed in background.
    \end{itemize}
  \end{block}
  \vspace*{-1ex}
  \begin{block}{Discovering your \colbf{jobs}}
    \texttt{jobs}
    \begin{itemize}
      \item Prints the running, paused, or recently stopped jobs.
      \item Prints jobs with their \texttt{JOB ID}s.
    \end{itemize}
  \end{block}
\end{frame}

\begin{frame}[fragile]{Dealing with Excess Output}
  \begin{itemize}[<+- | alert@+>]
    \item Many programs output continuously as they run.  Try \texttt{vlc}.  Pretty, but also annoying.
    \item Redirect the output!
    \item Saving the output:
    \begin{block}{Save \texttt{ping} results}
      \texttt{ping google.com > testping.log \&}
      \begin{itemize}
        \item A \texttt{.log} file is common.
        \item Note you need to eventually end this \texttt{ping}!
      \end{itemize}
    \end{block}
    \item Ignoring the output:
    \begin{block}{Ignoring it all}
      \texttt{vlc best\_song\_ever.flac -- \&> /dev/null \&}
    \end{block}
  \end{itemize}
\end{frame}
%%%%%
%%%%%
%%%%%
%%%%%
%%%%% section jobs
%%%%
%%%
%%
%

%
%%
%%%
%%%%
%%%%% section job_control_demo
%%%%%
%%%%%
%%%%%
%%%%%
\section{Job Control Demo}
\label{sec:job_control_demo}

\begin{frame}[fragile]{Controlling Jobs}
  I did a demo on-the-fly in class demonstrating job control, pausing, resuming, etc.  I encourage you to follow
  the \emph{ex post facto} demo here:

  \begin{center}
    \href{https://github.com/cs2043-sp16/lecture-demos/tree/master/lec07}{https://github.com/cs2043-sp16/lecture-demos/tree/master/lec07}
  \end{center}
\end{frame}

%%%%%
%%%%%
%%%%%
%%%%%
%%%%% section job_control_demo
%%%%
%%%
%%
%

\begin{frame}[allowframebreaks]{References}
  \bibliography{references}
  \bibliographystyle{abbrv}
\end{frame}

\end{document}
