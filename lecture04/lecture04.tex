% Cornell CS 2043: Unix Tools and Scripting
% Spring 2016
%
% http://cs2043-sp16.github.io/
%
% These slides have been created using the metroplis theme demo found here:
%
%     https://github.com/matze/mtheme.git
%
% I am a huge fan of the metropolis theme...thanks @matze!

\documentclass[11pt]{beamer}
\usetheme{metropolis}
\metroset{background=dark} % easier on the eyes, in my opinion
\setbeamertemplate{bibliography item}{\insertbiblabel}

% linking
\usepackage{hyperref}

% Nicer tables
\usepackage{booktabs}
\usepackage[scale=2]{ccicons}

% finer control over enumerations
% example:
%   \begin{enumerate}[(1)]
%     \item first
%     \item second
%     \item ...
%   \end{enumerate}
%
%   and also
%
%   \begin{enumerate}[a)]
%   \begin{enumerate}[i.]
%   \begin{enumerate}[$\mathbb{R}$\textsuperscript{\arabic{enumi}}]
%
% and so on
\usepackage{enumerate}

%
% sultry code listings
%
\usepackage{minted}
\BeforeBeginEnvironment{minted}{\begin{block}{}} % These surround minted code with beamer blocks
\AfterEndEnvironment{minted}{\end{block}}        % to accentuate them

%
% Colors thanks to various themes from Monokai: http://www.colourlovers.com/lover/Monokai
%
\definecolor{make-it-concrete}{rgb}{0.670588, 0.666667, 0.596078}      % (171, 170, 152) / 255
\definecolor{cheap-carpet}{rgb}{0.145098, 0.196078, 0.200000}% (37,50,51) / 255
\definecolor{hairball}{rgb}{0.403922, 0.411765, 0.380392}% (103,105,97) / 255

%
% General slide coloring
%
\setbeamercolor{frametitle}{% adjusts too-white colors from beamer-metropolis
  bg=hairball!66,
  fg=cheap-carpet
}
\setbeamercolor{normal text}{
  bg=mDarkTeal,   % note: defined by beamer-metropolis
  fg=hairball!44
}
%
% beamer-block shaping and colors
%
\setbeamertemplate{blocks}[rounded]
\setbeamercolor{block title}{
  bg=make-it-concrete!66,
  fg=cheap-carpet
}
\setbeamercolor{block body}{
  bg=cheap-carpet,
  fg=make-it-concrete!66
}

%
% Custom commands
%
\newcommand{\colbf}[1]{\textcolor{mLightBrown!77!black}{#1}}% bold with color; mLightBrown defined by beamer-metropolis
\newcommand{\tsub}{\textsubscript}                          % more compact
\newcommand{\tsup}{\textsuperscript}                        % " " " " " "
\def\wl{\par \vspace{\baselineskip}}                        % force newline; MANY thanks:
                                                            %   https://groups.google.com/d/msg/comp.text.tex/abxw5nTALzg/puXHnyx6pM8J

%%%%%%%%%%%%%%%%%%%%%%%%%%%%%%%%%%%%%%%%%%%%%%%%%%%%%%%%%%%%%%%%%%%%%%%%%%%%%%%%%%%%%%%%%%%%%%%%%%%%%%%%%%%%%%%
%%%%%%%%%%%%%%%%%%%%%%%%%%%%%%%%%%%%%%%%%%%%%%%%%%%%%%%%%%%%%%%%%%%%%%%%%%%%%%%%%%%%%%%%%%%%%%%%%%%%%%%%%%%%%%%
%%%%%%%%%%%%%%%%%%%%%%%%%%%%%%%%%%%%%%%%%%%%%%%%%%%%%%%%%%%%%%%%%%%%%%%%%%%%%%%%%%%%%%%%%%%%%%%%%%%%%%%%%%%%%%%
%%%%%%%%%%%%%%%%%%%%%%%%%%%%%%%%%%%%%%%%%%%%%%%%%%%%%%%%%%%%%%%%%%%%%%%%%%%%%%%%%%%%%%%%%%%%%%%%%%%%%%%%%%%%%%%
%
% \begin
%
%%%%%%%%%%%%%%%%%%%%%%%%%%%%%%%%%%%%%%%%%%%%%%%%%%%%%%%%%%%%%%%%%%%%%%%%%%%%%%%%%%%%%%%%%%%%%%%%%%%%%%%%%%%%%%%
%%%%%%%%%%%%%%%%%%%%%%%%%%%%%%%%%%%%%%%%%%%%%%%%%%%%%%%%%%%%%%%%%%%%%%%%%%%%%%%%%%%%%%%%%%%%%%%%%%%%%%%%%%%%%%%
%%%%%%%%%%%%%%%%%%%%%%%%%%%%%%%%%%%%%%%%%%%%%%%%%%%%%%%%%%%%%%%%%%%%%%%%%%%%%%%%%%%%%%%%%%%%%%%%%%%%%%%%%%%%%%%
%%%%%%%%%%%%%%%%%%%%%%%%%%%%%%%%%%%%%%%%%%%%%%%%%%%%%%%%%%%%%%%%%%%%%%%%%%%%%%%%%%%%%%%%%%%%%%%%%%%%%%%%%%%%%%%
\title{04 \-- More Files, Chaining Commands, and your First(?) Git Repository}
\subtitle{CS 2043: Unix Tools and Scripting, Spring 2016 \cite{prevSemesters}}
\date{February 3rd, 2016}
\author{Stephen McDowell}
\institute{Cornell University}

\begin{document}
\maketitle

% TOC
\begin{frame}{Table of contents}
  \setbeamertemplate{section in toc}[sections numbered]
  \tableofcontents[hideallsubsections]
\end{frame}

\begin{frame}{Some Logistics}
  \begin{itemize}[<+- | alert@+>]
    \item Last day to add is today.
    \item (Poll) The demo last time
  \end{itemize}
\end{frame}

%
%%
%%%
%%%%
%%%%% section recap_on_permissions
%%%%%
%%%%%
%%%%%
%%%%%
\section{Recap on Permissions}
\label{sec:recap_on_permissions}

\begin{frame}[fragile]{The Octal Version of \texttt{chmod}}
  Last time I linked you to this\cite{chmod} website for a good explanation.  For the formula hungry, you can
  represent \texttt{r}, \texttt{w}, and \texttt{x} as binary variables (where 0 is off, and 1 is on).  Then
  the formula for the modes is

  $$
    r\cdot2^2 + w\cdot2^1 + x\cdot2^0
  $$

  \begin{block}{Examples}
    \begin{itemize}
      \item \texttt{chmod 755}: \texttt{rwxr-xr-x}
      \item \texttt{chmod 777}: \texttt{rwxrwxrwx}
      \item \texttt{chmod 600}: \texttt{rw-------}
    \end{itemize}
  \end{block}

  If that makes less sense to you, feel free to ignore it.
\end{frame}

\begin{frame}[fragile]{Super Confused...}
  \begin{block}{\colbf{S}uper\colbf{u}ser \colbf{Do}}
    \texttt{sudo <command>}
    \begin{itemize}
      \item Execute \texttt{<command>} as the super user
      \item The regular user (e.g. \texttt{student}) is executing the \texttt{sudo} command, \emph{not} the root
      \item You enter your user password
      \item You can only execute \texttt{sudo} if you are an "administrator"\tsup{*}
    \end{itemize}
  \end{block}

  So on the course VMs the \texttt{student} user originally had the password \texttt{student}, so that is what
  you would type if you were executing \texttt{sudo}.  On your personal Mac (or native Linux install), you would
  be typing whatever your password is to login to the computer.

  {\tiny \tsup{*}Note that where you look to see who can execute \texttt{sudo} varies greatly between distributions.  It may be
    easier to just try it...you'll know whether you have access very quickly.}
\end{frame}

\begin{frame}[fragile]{Super Confused...}
  If you know the root password, then you can become root using \texttt{su} directly.
  \begin{block}{\colbf{S}witch \colbf{U}ser}
    \texttt{su <user\_name>}
    \begin{itemize}
      \item Switches to user \texttt{user\_name}
      \item The password you enter is the password for \texttt{user\_name}
      \item If no username is specified, \texttt{root} is implied
    \end{itemize}
  \end{block}

  So the commands \texttt{sudo su root} and \texttt{sudo su} are equivalent, but since you typed \texttt{sudo}
  first that is why you type the user password.  If you just execute \texttt{su} directly, then you have to type
  the \texttt{root} password.
\end{frame}

\begin{frame}[fragile]{Default Permissions}
  When you create files during a particular session, the mode you are running in determines what the permissions
  will be.  Changing the \texttt{umask} only applies for the remainder of the session (e.g. until you close
  the terminal window you were writing this in).  If you understand what this means, it is just a bit mask with
  \texttt{0o777}.

  \begin{block}{\colbf{U}ser \colbf{mask}}
    \texttt{umask <mode>}
    \begin{itemize}
      \item Remove \texttt{mode} from the file's permissions
      \item Similar syntax to \texttt{chmod}
      \begin{itemize}
        \item \texttt{umask 077}: full access to the user, no access to anybody else
        \item \texttt{umask g+w}: enables group write permissions
      \end{itemize}
      \item \texttt{umask -S}: display the current mask
    \end{itemize}
  \end{block}
\end{frame}

%%%%%
%%%%%
%%%%%
%%%%%
%%%%% section recap_on_permissions
%%%%
%%%
%%
%

%
%%
%%%
%%%%
%%%%% section file_compression
%%%%%
%%%%%
%%%%%
%%%%%
\section{File Compression}
\label{sec:file_compression}

\begin{frame}[fragile]{Making Archives: Zip}
  \begin{block}{\colbf{Zip}}
    \texttt{zip <name\_of\_archive> <files\_to\_include>}
    \begin{itemize}
      \item Note I said \emph{files}.
      \begin{itemize}
        \item E.g. \texttt{zip files.zip a.txt b.txt c.txt}
        \item These will extract to \texttt{a.txt}, \texttt{b.txt}, and \texttt{c.txt} in the current directory
      \end{itemize}
        \item To do folders, you need recursion
      \begin{itemize}
        \item \texttt{zip -r folder.zip my\_files/}
        \item This will extract to a folder named \texttt{my\_files}, with whatever was inside of it in tact
      \end{itemize}
    \end{itemize}
  \end{block}
  \begin{block}{\colbf{Unzip}}
    \texttt{unzip <archive\_name>}
  \end{block}
  \colbf{Note}: The original files DO stay in tact.
\end{frame}

\begin{frame}{Making Archives: Gzip}
  \begin{block}{\colbf{Gzip}}
    \texttt{gzip <files\_to\_compress>}
    \begin{itemize}
      \item \texttt{--fast}: less time to compress, larger file
      \item \texttt{--best}: more time to compress, smaller file
      \item Read the \texttt{man} page, lots of options
    \end{itemize}
  \end{block}
  \begin{block}{\colbf{Gunzip}}
    \texttt{gunzip <archive\_name>}
  \end{block}
  \colbf{Notes}:
  \begin{itemize}
    \item By default, \emph{replaces} the original files!
    \begin{itemize}
      \item You can use \texttt{--keep} to bypass this.
    \end{itemize}
    \item Does not bundle the files.
    \item Usually has better compression than \texttt{zip}.
  \end{itemize}
\end{frame}

\begin{frame}[fragile]{Making Archives: Tar}
  \begin{block}{\colbf{T}ape \colbf{Ar}chive}
    \texttt{tar -cf <tar\_archive\_name> <files\_to\_compress>}
    \begin{itemize}
      \item \colbf{C}reate a tar archive
    \end{itemize}
    \texttt{tar -xf <tar\_archive\_name>}
    \begin{itemize}
      \item E\colbf{x}tract all files from archive
    \end{itemize}
  \end{block}
  \colbf{Notes}:
  \begin{itemize}
    \item \texttt{tar} is just a bundling suite, creating a single file.
    \item By default, it does \emph{not} compress.
    \item Original files DO stay in tact.
    \item Unlike \texttt{zip}, you do not need the \texttt{-r} flag for folders :)
  \end{itemize}
\end{frame}

\begin{frame}[fragile]{Making Archives: Tarballs}
  \begin{block}{Making tarballs}
    \texttt{tar -c(z/j)f <archive\_name> <source\_files>}\\
    \texttt{tar -x(z/j)f <archive\_name>}
    \begin{itemize}
      \item The \texttt{-z} flag specifies \texttt{gzip} as the compression method
      \begin{itemize}
        \item Extension convention: \texttt{.tar.gz}
        \item YOU have to specify this, e.g. \texttt{tar -cjf files.tar.gz files/}
      \end{itemize}
      \item The \texttt{-j} flag specifies \texttt{bzip2} as the compression method
      \begin{itemize}
        \item Extension convention: \texttt{.tar.bz2}
        \item YOU have to specify this, e.g. \texttt{tar -cjf files.tar.bz2 files/}
      \end{itemize}
    \end{itemize}
  \end{block}
  \colbf{Note}:
  \begin{itemize}
    \item Extraction can \emph{usually} happen automatically:
    \begin{itemize}
      \item \texttt{tar -xf files.tar.gz} will usually work (no \texttt{-z})
    \end{itemize}
  \end{itemize}
\end{frame}

%%%%%
%%%%%
%%%%%
%%%%%
%%%%% section file_compression
%%%%
%%%
%%
%

%
%%
%%%
%%%%
%%%%% section assorted_commands
%%%%%
%%%%%
%%%%%
%%%%%
\section{Assorted Commands}
\label{sec:assorted_commands}

\begin{frame}[fragile]{Before we can Chain...}
  ...we need some more interesting tools to chain together!
\end{frame}

\begin{frame}[fragile]{Counting}
  \begin{block}{\colbf{W}ord \colbf{C}ount}
    \texttt{wc [options] <file>}
    \begin{itemize}
      \item \texttt{-l}: count the number of lines
      \item \texttt{-w}: count the number of words
      \item \texttt{-m}: count the number of characters
      \item \texttt{-c}: count the number of bytes
    \end{itemize}
  \end{block}
  Great for things like
  \begin{itemize}[<+- | alert@+>]
    \item revelling in the number of lines you have programmed
    \item analyzing the verbosity of your personal statement
    \item showing people how cool you are
  \end{itemize}
\end{frame}

\begin{frame}[fragile]{Sorting}
  \begin{block}{\colbf{Sort}}
    \texttt{sort [options] <file>}
    \begin{itemize}
      \item Default: by \texttt{ASCII} code (roughly alphabetical) for the whole line
      \item \texttt{-r}: reverse order
      \item \texttt{-n}: numerical order
      \item \texttt{-u}: remove duplicates
    \end{itemize}
  \end{block}
  \vspace*{-2ex}
  \begin{minipage}[t]{\textwidth}
    \begin{minipage}[t]{.3\textwidth}
      \begin{minted}[bgcolor=bg, gobble=8, fontsize=\fontsize{7.5}{0}]{bash}
        > cat peeps.txt
        Manson, Charles
        Bundy, Ted
        Bundy, Jed
        Nevs, Sven
        Nevs, Sven
      \end{minted}
    \end{minipage}
    \hfill
    \begin{minipage}[t]{.3\textwidth}
      \begin{minted}[bgcolor=bg, gobble=8, fontsize=\fontsize{7.5}{0}]{bash}
        > sort -r peeps.txt
        Nevs, Sven
        Nevs, Sven
        Manson, Charles
        Bundy, Ted
        Bundy, Jed
      \end{minted}
    \end{minipage}
    \hfill
    \begin{minipage}[t]{.3\textwidth}
      \begin{minted}[bgcolor=bg, gobble=8, fontsize=\fontsize{7.5}{0}]{bash}
        > sort -ru peeps.txt
        Nevs, Sven
        Manson, Charles
        Bundy, Ted
        Bundy, Jed
        # only 1 Nevs, Sven
      \end{minted}
    \end{minipage}
  \end{minipage}
\end{frame}

\begin{frame}[fragile]{Advanced Sorting}
  The \texttt{sort} command is quite powerful, for example you can do
  \begin{itemize}
    \item \texttt{sort -n -k 2 -t "," <filename>}
    \begin{itemize}
      \item sorts the file numerically by using the second column, separating by a comma as the separator instead
            of a space
    \end{itemize}
    \item read the \texttt{man} page!
  \end{itemize}
  \vspace*{-2ex}
  \begin{minipage}[t]{\textwidth}
    \begin{minipage}[t]{.48\textwidth}
      \begin{minted}[bgcolor=bg, gobble=8, fontsize=\fontsize{7.5}{0}]{bash}
        > cat numbers.txt
        02,there
        04,how
        01,hi
        06,you
        03,bob
        05,are
      \end{minted}
    \end{minipage}
    \hfill
    \begin{minipage}[t]{.48\textwidth}
      \begin{minted}[bgcolor=bg, gobble=8, fontsize=\fontsize{7.5}{0}]{bash}
        > sort -n -k 2 -t "," numbers.txt
        01,hi
        02,there
        03,bob
        04,how
        05,are
        06,you
      \end{minted}
    \end{minipage}
  \end{minipage}
\end{frame}

\begin{frame}[fragile]{Special Snowflakes}
  \begin{block}{\colbf{uniq}}
    \texttt{uniq [options] <file>}
    \begin{itemize}
      \item No flags: discards all but one of successive identical lines
      \item \texttt{-c}: prints the number of successive identical lines next to each line
    \end{itemize}
  \end{block}
\end{frame}

\begin{frame}[fragile]{Search and Replace}
  \begin{block}{\colbf{Tr}anslate}
    \texttt{tr [options] <set1> [set2]}
    \begin{itemize}
      \item Translate or delete characters
      \item Sets are strings of characters
      \item By default, searches for strings matching \texttt{set1} and replaces them with \texttt{set2}
      \item You can use regular expressions (we'll get there soon!)
    \end{itemize}
  \end{block}

  The \texttt{tr} command only works with streams.  There will be some examples of these after we learn about
  chaining commands in the next section.
\end{frame}

%%%%%
%%%%%
%%%%%
%%%%%
%%%%% section assorted_commands
%%%%
%%%
%%
%

%
%%
%%%
%%%%
%%%%% section chaining_commands
%%%%%
%%%%%
%%%%%
%%%%%
\section{Chaining Commands}
\label{sec:chaining_commands}

\begin{frame}[fragile]{Your Environment and Variables}
  There are various environment variables defined in your environment.  The are almost always all capital letters,
  and you obtain their value by dereferencing them with a \texttt{\$}.  Some examples

  \begin{minted}[bgcolor=bg, gobble=4, fontsize=\footnotesize]{bash}
    > echo $PWD    # present working directory
    > echo $OLDPWD # print previous working directory
    > printenv     # print all defined environment variables
  \end{minted}

  It turns out, when you execute commands they have something called an "exit code".  The exit code of the last
  command executed is stored in the \texttt{\$?} environment variable.
\end{frame}

\begin{frame}[fragile]{What is Defined?}
  \begin{itemize}[<+- | alert@+>]
      \item The environment
      \begin{itemize}[<+- | alert@+>]
        \item \texttt{env}: displays all environment variables
        \item \texttt{unsetenv <name>}: remove an environment variable
      \end{itemize}
      \item The local variables
      \begin{itemize}[<+- | alert@+>]
        \item \texttt{set}: displays all shell / local variables
        \item \texttt{unset <name>}: remove a shell variable
      \end{itemize}
      \item We'll cover these a little more when we talk about customizing your terminal shell
  \end{itemize}
\end{frame}

\begin{frame}[fragile]{Exit Codes}
  There are various exit codes, here are a few examples:
  \begin{minted}[bgcolor=bg, gobble=4, fontsize=\footnotesize]{bash}
    > super_awesome_command
    bash: super_awesome_command: command not found...
    > echo $?
    127
    > echo "What is the exit code we want?"
    > echo $?
    0
  \end{minted}
  \begin{itemize}[<+- | alert@+>]
    \item The success code we want is actually \texttt{0}.  Refer to \cite{exit} for some more examples.
    \item Remember that \texttt{cat /dev/urandom} trickery?  You will have to \texttt{ctrl+c} to kill it, what
          would the exit code be?
  \end{itemize}
\end{frame}

\begin{frame}[fragile]{Executing Multiple Commands in a Row}
  With exit codes, we can define some simple rules to chain commands together:

  \begin{itemize}[<+- | alert@+>]
    \item Always execute:

    \begin{minted}[bgcolor=bg, gobble=6, fontsize=\footnotesize]{bash}
      > cmd1; cmd2   # exec cmd1 first, then cmd2
    \end{minted}

    \item Execute conditioned upon exit code:

    \begin{minted}[bgcolor=bg, gobble=6, fontsize=\footnotesize]{bash}
      > cmd1 && cmd2 # exec cmd2 only if cmd1 returned 0
      > cmd1 || cmd2 # exec cmd2 only if cmd1 returned NOT 0
    \end{minted}

    \item Kind of backwards, in terms of what means continue for \emph{and}, but that was likely easier to implement
          since there is only one \texttt{0} and many \colbf{not} \texttt{0}'s.
  \end{itemize}
\end{frame}

\begin{frame}[fragile]{Piping Commands}
  Bash scripting is all about combining simple commands together to do more powerful things.  This is accomplished
  using the "pipe" character.

  \begin{block}{Piping}
    \texttt{<command1> | <command2>}
    \begin{itemize}
      \item Passes the output from \texttt{command1} to be the input of \texttt{command2}
      \item Works for \emph{heaps} of programs that take input and provide output to the terminal.
    \end{itemize}
  \end{block}
\end{frame}

\begin{frame}[fragile]{Some Piping Examples}
  \begin{block}{Piping along...}
    \texttt{ls -al /bin | less}
    \begin{itemize}
      \item Allows you to scroll through the long list of programs in \texttt{/bin}
    \end{itemize}
    \texttt{history | tail -20 | head -10}
    \begin{itemize}
      \item Displays the 10\tsup{th} \-- 19\tsup{th} previous commands from the previous session
    \end{itemize}
    \texttt{echo * | tr ' ' '\textbackslash n'}
    \begin{itemize}
      \item Replaces all spaces characters with new lines
    \end{itemize}
  \end{block}
\end{frame}

\begin{frame}[fragile]{Redirection}
  To redirect input / output streams, you can use one of \texttt{>}, \texttt{>>}, \texttt{<}, or \texttt{<<}.
  \begin{itemize}[<+- | alert@+>]
    \item to redirect standard output, use the \texttt{>} operator
    \begin{itemize}[<+- | alert@+>]
      \item \texttt{command > file}
    \end{itemize}
    \item to redirect standard input, use the \texttt{<} operator
    \begin{itemize}[<+- | alert@+>]
      \item \texttt{command < file}
    \end{itemize}
    \item to redirect standard error, use the \texttt{>} operator and specify the stream number \texttt{2}
    \begin{itemize}[<+- | alert@+>]
      \item \texttt{command 2> file}
    \end{itemize}
    \item combine streams together by using \texttt{2>\&1}
    \begin{itemize}
      \item This says: send standard error to where standard output is going.
      \item Useful for debugging / catching error messages...
      \item ...or ignoring them (you will often see that sent to \texttt{/dev/null})
    \end{itemize}
  \end{itemize}
\end{frame}

\begin{frame}[fragile]{Redirection Example}
  Bash processes I/O redirection from left to right, allowing us to do fun things like this:

  \begin{block}{Magic}
    \texttt{tr -cd '0-9' < test1.txt > test2.txt}
    \begin{itemize}
      \item deletes everything but the numbers from \texttt{test1.txt}, then store them in \texttt{test2.txt}
    \end{itemize}
  \end{block}

  Piping and Redirection are quite sophisticated, please refer to the Wikipedia page in \cite{wiki}.
\end{frame}
%%%%%
%%%%%
%%%%%
%%%%%
%%%%% section chaining_commands
%%%%
%%%
%%
%

%
%%
%%%
%%%%
%%%%% section more_git_forking_a_repository
%%%%%
%%%%%
%%%%%
%%%%%
\section{More Git: Forking a Repository}
\label{sec:more_git_forking_a_repository}

\begin{frame}[fragile]{In class demo...}
  \href{https://github.com/cs2043-sp16/lecture-demos/tree/master/lec04}{https://github.com/cs2043-sp16/lecture-demos/tree/master/lec04}
\end{frame}

%%%%%
%%%%%
%%%%%
%%%%%
%%%%% section more_git_forking_a_repository
%%%%
%%%
%%
%

\begin{frame}[allowframebreaks]{References}
  \bibliography{lecture04/references}
  \bibliographystyle{abbrv}
\end{frame}

\end{document}
